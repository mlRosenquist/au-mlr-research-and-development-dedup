\documentclass[conference]{IEEEtran}
\IEEEoverridecommandlockouts
\usepackage{graphicx}
\usepackage[colorlinks = true,
            linkcolor = black,
            urlcolor  = blue,
            citecolor = black,
            anchorcolor = blue]{hyperref}
\usepackage{amsmath,amssymb,amsfonts, mathtools}
\usepackage{algorithmic}
\usepackage{textcomp}
\usepackage{lipsum}                     
\usepackage{xargs}                      
\usepackage[pdftex,dvipsnames, table]{xcolor}  
\usepackage[table]{xcolor}
\usepackage{float}
\usepackage{subcaption}
\usepackage{stfloats}
\usepackage{bbm}
\usepackage{array, rotating}
\usepackage{multirow, makecell}
\usepackage{fmtcount}
\usepackage{lipsum}
\usepackage{mwe}
\usepackage{hhline}

\usepackage{boldline}
\usepackage{array}
\newcolumntype{?}{!{\vrule width 1pt}}

\usepackage[
backend=biber,
style=numeric,
sorting=none
]{biblatex}
\addbibresource[]{misc/references.bib}


\rowcolors{2}{gray!10}{white}

\newcommand{\titlecol}{\cellcolor{gray!30}}

% 
\usepackage[colorinlistoftodos,prependcaption,textsize=tiny]{todonotes}
\newcommandx{\unfinished}[2][1=]{\todo[linecolor=red,backgroundcolor=red!25,bordercolor=red,#1]{#2}}
\newcommandx{\change}[2][1=]{\todo[linecolor=blue,backgroundcolor=blue!25,bordercolor=blue,#1]{#2}}
\newcommandx{\info}[2][1=]{\todo[linecolor=OliveGreen,backgroundcolor=OliveGreen!25,bordercolor=OliveGreen,#1]{#2}}
\newcommandx{\improvement}[2][1=]{\todo[linecolor=Plum,backgroundcolor=Plum!25,bordercolor=Plum,#1]{#2}}
\newcommandx{\thiswillnotshow}[2][1=]{\todo[disable,#1]{#2}}

\begin{document}
\title{Outlier Detection of Generalized Deduplication Compressed Data\\
}

\author{\IEEEauthorblockN{Morten Lyng Rosenquist}
  \IEEEauthorblockA{\textit{Faculty of Technical Sciences} \\
    \textit{Aarhus University}\\
    Aarhus, Denmark \\
    201706031 \\ \\
    \today
  }
}

\maketitle
\thispagestyle{plain}
\pagestyle{plain}
\begin{abstract}
  The emerging growth of Internet of Things leads to an immense volume of data being generated. Leading to numerous network and storage challenges that are to be solved with compression. Decompressing the large amount of data to apply analytical methods is costly. Generalized Deduplication is a compression technique that has low distortion and enables analytical capabilities of compressed data. Anomaly detection techniques are applied and evaluated on the compressed and decompressed data. The techniques used are Isolation Forest and an extended version tailored to data compressed by Generalized Deduplication. It showed that the capability of identifying anomalies is not lost after compression. The extended version succeeding in classifying fewer inliers as outliers, but at the cost of processing time and memory usage.       
\end{abstract}

\begin{IEEEkeywords}
  % Buzzwords regarding field of work
  % 3-8 words
  Internet of Things, Anomaly Detection, data compression, data mining
\end{IEEEkeywords}

\section{Introduction}
\unfinished{write introduction}
% Notes: 
% Introduce the subject
% What does the paper research
% What, how and why?
% Relation to IoT 
% Start broad and narrow it down to what will be looked at in paper
% Write later
% Faster computation, maintaining precision
% Operate on smaller data
% Only transmit bases and counts 
% Motivate that stuff can be done on edge 

% -----------

% What to include:

\section{Background}\label{sec:background}
\unfinished{Write background}

\subsection{Generalized Deduplication}
Deduplication is a technique to perform compression in storage systems. The technique works by utilizing the simarlity of file chunks. Each unique file chunk is stored once. Subsequent copies of the chunks are then replaced with a reference to the stored chunk. The method is established and shown to have good compression gain on various practical scenarios \cite{deduplication}. However, if there are minor discrepancies in the file chunks, the technique will not leverage any of the similarities. Resulting in the near-identical chunks being stored in full. Sensor data from IoT devices is one example of the data potentially being near-identical. 

To utilize the similarities in the almost identical data, a generalization of deduplication has been studied.     
This method consider the chunks at the bit level and splits them into two parts, the \textit{base} and \textit{deviation}. The \textit{base} is the identical part that is to be stored once and herafter referenced with pointers. The \textit{deviation} is the disparity between the chunks. Looking at a simple example with four 6-bit numbers, $100000$, $100001$, $100010$ and $100011$. It can be identified that the four most significant bits of the numbers are identical. Hence, leading to all having a shared \textit{base} of $1000$. The two least significant bits are then the \textit{deviation}\cite{gen-deduplication}.   

\subsection{Isolation Forest}
Anomaly detection is a combination of outlier- and novelty detection. Including both identifying outliers in the training data and determining if unseen observations are outliers. Isolation Forest (iForest) is an anomaly detection method. It differs from other popular techniques in the way that it identifies anomalies explicitly instead of profiling ordinary data points\cite{iforest}. IForest utilizes decision trees similar to other tree ensemble methods.
The main principle is to recursively split each data point, and then evaluate the amount of splits necessary to split each data point. The logic is that anomalies will requires less splits to be isolated than an ordinary point.  
Trees are built by selecting a random feature and then selecting a random value between the minimum and maximum value of that feature. The process is then repeated untill all data points are isolated or a maximum height of the tree is reached. An illustration can be seen on figure.... \todo{Tilføj figur med plot af data points og tilhørende træ.}

\section{Methods}\label{sec:methods}
\unfinished{write method}
% Notes: 
% Here we describe how the work was done.
% Mathematical description of new model. Maybe pseudocode of algorithm? 
% Discussion of new model? Performance, complexicity, etc? 
% We describe the extended isolation forest here. In Direct Analysis of GD .. the used method has its own section. 
% Concept, and new stuffs
% Running isolation as is directly on bases - phase 1
% Running extended isolation on the bases  - phase 2
% Can include or exclude counts 
% tradeoffs: counts, precision. 

% -----------

% What to include:

\section{Experiments}\label{sec:experiments}
\unfinished{write experiments}

% Notes: 
% Should this be split in 2 as usual? In direct analytics of GD... there is a "performance of..." section. Is that better? 
% Description of experiements  
% Describe dataset 
% Suboptimal extention of sklearn
% Performance: time, f1, recall, precision, accuracy. 
% Comparisons (quantitative and qualitative)
% Illustration of performance (Precision-Recall, ROC, confusion matrix)

% -----------

% What to include:
% - datasets
% - procedure
% - what is measured



\section{Results}\label{sec:results}
This section will present and interpret the results from the conducted experiments. There are many experimental setups. Three different models evaluated on five datasets with varying amount of compression. Then in addition to that, there is many metrics to be looked at. To illustrate this comprehensive amount of results, Table \ref{tab:comparison} has been constructed. The Table contains the results of three of the five datasets. The Table will be reffered to repeatedly in the following examination of the results. The values that are reffered to is marked with bold.

\subsection{Execution time}
Starting off with looking at the execution time. The training and test time on the synthetic dataset can visually be seen on Figure \ref{fig:performance_time}. It shows no apparent difference between neither the training time or testing of the original and bases model. However, there is a clear difference between those two and DupRes. Both the training in which the uniques are found and in the testing in which the duplicates are looked up adds additional execution time. This performance cost can also clearly be seen in the table for the pendigits dataset. Here we have 93.7/55.3 ms train/test time for the original model, while DupRes with 1 deviation bit has 135/282 ms train/test time. Additionally, it can be seen that DupRes with 6 deviation bits has 145/131 train/test time. This implies that, as the amount of deviation bits rise the testing time decreases. This makes sense as less uniques are found during the training phase, and thus less entries in the table need to be looked through during testing. This incentivize to not use the model if a low amount of duplicates is expected in the data or a low amount of deviation bits is utilized. Originally an implementation utilizing pandas'\cite{pandas} dataframes was used. However, this performed notably worse then the current implementation which uses numpy\cite{numpy} arrays. To optimize this further an implementation could be made in c or c++ and utilize the interoperability to Python.

\begin{figure}
  \centering
  \includegraphics[width=0.8\linewidth]{images/performance_time.png}
  \caption{}
  \label{fig:performance_time}
\end{figure}

\subsection{Metrics and Scoring}
Then we look at the performance of the models in terms of metrics and scoring. Figure \ref{fig:performance_metrics} illustrates the performance metrics of the models on the datasets. In general it can be seen on the figure that the models are performing quite similar. Investigating the metrics for the synthetic dataset a clear pattern can be seen. The bases model performs better than the original, meanwhile DupRes performs better than the bases. The cause of this can be derived from the recall. Since the positives are the inliers the original is simply classifying more inliers as outliers than the others. It is using too few splits in isolating them. As we compress the data the inliers are harder to isolate due to clustering. Utilizing the many duplicates in DupRes we are then classifiying less inliers as outliers. Similar trends is not be seen on the other datasets. On the recall of the pendigit dataset it is seen that the original model predicts less false positives than the others.

Looking at the table there is several interesting observations. All models except two detect every outlier. The two instances are the bases model and DupRes on the synthetic dataset with 6 deviation bits. The bases model predicts every observation as an outlier resulting in 0 scores. Since a lot of data points is grouped with this amount of deviation bits, DupRes will not do the same action as the other model, and therefore have a precision score of 0.98. This implies that one outlier observation might have grouped with the inliers. DupRes will wrongfully push this to be an inlier. Investigating the performance regarding the WBC set it is seen that the performance is almost identical. There are minor discrepancies in the accuracy however the rest is identical. This implies that even though compression is performed, the analytical capabilities is still intact. This is a general trend across all datasets, as was also depicted on Figure \ref{fig:performance_metrics}.

The amount of deviation bits shows a pattern. Looking at the Pendigits dataset the higher the amount of deviation bits the more inliers will be classified as an outlier. This is natural as the compression will place the observations in larger bins that are easier to separate. DupRes does not classify these inliers as outlier as the large binning has resulted in more duplicates and therefore pushing their score towards being an inlier.

\begin{figure}
  \centering
  \includegraphics[width=\linewidth]{images/performance_metrics.png}
  \caption{}
  \label{fig:performance_metrics}
\end{figure}
\subsection{Compression and Memory}
The two right-most column in the table describes the compression rate and memory accesed for each model. As the features of all datasets is 8-bit numbers the compression rate is quite simple to calculate: $c_{rate}=\frac{d}{8}$, where $d$ is the amount of deviation bits. The original model naturally has no compression. The two others compression varies on the their amount of deviation bits. $d=6$ for both of the models catches all outliers except on the synthetic data set as described previously. This is a quite significant compression rate of 75\%. Memory Accessed describe how much of the original data that is accesed. It is kind of the reverse of the compression rate. Most interestingly is the DupRes which also has the count table. This is to illustrate that it needs to store the table of uniques and their counts in memory. This shows the tradeoff with DupRes and performing original Isolation Forest on the bases. They have the same compression rate. However, DupRes needs additional memory to store the table. As seen in the results, the training time spent creating the table and the testing time looking up in the table is not to be neglected.       

\begin{table*}[t]
  \centering\sffamily
  \renewcommand{\theadfont}{\normalsize\bfseries}
  \setcellgapes{1ex}\makegapedcells
  \begin{tabular}{*{10}{|c}|c|}
    \hline
    \multirowthead{2}{Model} & \multirowthead{2}{Dataset} & \multirowthead{2}{Dev. bits} & \multicolumn{4}{c|}{\bfseries Metric} & \multicolumn{2}{c|}{\bfseries Time} & \multirowthead{2}{Compression Rate} & \multirowthead{2}{Memory Accessed}                                                                   \\
    \cline{4-9}
                             &                            &                              & \textbf{acc.}                         & \textbf{f1}                         & \textbf{rec.}                       & \textbf{prec.}                     & \textbf{T(ms)} & \textbf{E(ms)} &                               \\
    \hline
    Original                 & Synthetic                  &                              & 0.75                                  & 0.85                                & 0.73                                & 1                                  & 78.8           & 51.4           & 0\%    & 100\%                \\

    \cline{2-11}
                             & Pendigits                  &                              & 0.54                                  & 0.70                                & 0.54                                & 1                                  & \textbf{93.7}  & \textbf{55.3}  & 0\%    & 100\%                \\

    \cline{2-11}
                             & WBC                        &                              & 0.94                                  & 0.97                                & 0.94                                & 1                                  & 75.2           & 22.0           & 0\%    & 100\%                \\
    \hline

    \multirow{9}{*}{Bases}   & \multirow{3}{*}{Synthetic} & 1                            & 0.72                                  & 0.83                                & 0.71                                & 1                                  & 80.2           & 51.2           & 12.5\% & 87.5\%               \\
    \cline{3-11}
                             &                            & 3                            & 0.90                                  & 0.95                                & 0.90                                & 1                                  & 74.6           & 46.4           & 37.5\% & 62.5\%               \\
    \cline{3-11}
                             &                            & 6                            & 0.04                                  & 0                                   & 0                                   & \textbf{0}                         & 74.3           & 45.0           & 75.0\% & 25.0\%               \\
    \cline{2-11}
                             & \multirow{3}{*}{Pendigits} & 1                            & \textbf{0.53}                                  & 0.70                                & 0.53                                & 1                                  & 95.1           & 55.7           & 12.5\% & 87.5\%               \\
    \cline{3-11}
                             &                            & 3                            & \textbf{0.51}                                  & 0.67                                & 0.51                                & 1                                  & 94.3           & 55.9           & 37.5\% & 62.5\%               \\
    \cline{3-11}
                             &                            & 6                            & \textbf{0.42}                                  & 0.59                                & 0.41                                & 1                                  & 95.3           & 56.0           & 75.0\% & 25.0\%               \\
    \cline{2-11}
                             & \multirow{3}{*}{WBC}       & 1                            & 0.94                                  & 0.97                                & 0.94                                & 1                                  & 75.3           & 21.6           & 12.5\% & 87.5\%               \\
    \cline{3-11}
                             &                            & 3                            & 0.94                                  & 0.97                                & 0.94                                & 1                                  & 75.6           & 21.6           & 37.5\% & 62.5\%               \\
    \cline{3-11}
                             &                            & 6                            & 0.95                                  & 0.97                                & 0.94                                & 1                                  & 74.8           & 21.4           & 75.0\% & 25.0\%               \\
    \hline
    \multirow{9}{*}{DupRes}  & \multirow{3}{*}{Synthetic} & 1                            & 0.87                                  & 0.93                                & 0.87                                & 1                                  & 90.4           & 59.7           & 12.5\% & 87.5\% + count table \\
    \cline{3-11}
                             &                            & 3                            & 0.92                                  & 0.95                                & 0.91                                & 1                                  & 85.8           & 55.4           & 37.5\% & 62.5\% + count table \\
    \cline{3-11}
                             &                            & 6                            & 0.98                                  & 0.99                                & 1.00                                & \textbf{0.98}                      & 86.5           & 53.4           & 75.0\% & 25.0\% + count table \\
    \cline{2-11}
                             & \multirow{3}{*}{Pendigits} & 1                            & \textbf{0.54}                                  & 0.70                                & 0.53                                & 1                                  & \textbf{135}   & \textbf{282}   & 12.5\% & 87.5\% + count table \\
    \cline{3-11}
                             &                            & 3                            & \textbf{0.50}                                  & 0.67                                & 0.50                                & 1                                  & 135            & 286            & 37.5\% & 62.5\% + count table \\
    \cline{3-11}
                             &                            & 6                            & \textbf{0.80}                                  & 0.89                                & 0.80                                & 1                                  & \textbf{145}   & \textbf{131}   & 75.0\% & 25.0\% + count table \\
    \cline{2-11}
                             & \multirow{3}{*}{WBC}       & 1                            & 0.94                                  & 0.97                                & 0.94                                & 1                                  & 86.9           & 23.4           & 12.5\% & 87.5\% + count table \\
    \cline{3-11}
                             &                            & 3                            & 0.94                                  & 0.97                                & 0.94                                & 1                                  & 87.7           & 23.7           & 37.5\% & 62.5\% + count table \\
    \cline{3-11}
                             &                            & 6                            & 0.95                                  & 0.97                                & 0.94                                & 1                                  & 85.6           & 23.5           & 75.0\% & 25.0\% + count table \\
    \hline
  \end{tabular}
  \caption{Table of stuff}
  \label{tab:comparison}
\end{table*}

\section{Conclusion}\label{sec:conclusion}
\unfinished{write conclusion}

\begin{itemize}
    \item Summary of the work
    \item Key Observations
    \item Limitation and future work ()
\end{itemize}

\section{Future work}
DupRes currently uses the $log_2(x_{count})$ term to tune the score towards being an inlier based on the count. This is potentially a hyperparameter that can be tuned. Maybe a completely different term fits better. This could be something that also includes the dimensionality of the data or other factors. This report looked at a limited amount of datasets. It would therefore be interesting to see how the different models perform on others. In regards to generalized deduplication we only investigated 8 bit integers. Thus, a task is to research the anomaly detection behaviour on floating numbers.      

\printbibliography[title={References}]

\end{document}